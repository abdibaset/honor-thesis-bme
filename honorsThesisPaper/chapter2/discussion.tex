This study demonstrates the feasibility of using mPAI and PAI with soluble PD1 in tissue mimicking phantoms to quantify the expression and availability of the immune checkpoint receptor, respectively. By modeling diffusion as a two-dimensional random walk and applying the mean squared displacement (MSD) relation, we were able to experimentally determine the diffusion coefficients of various fluorescently labeled antibodies and proteins and calculate the corresponding incubation times required for the imaging to diffuse through the 10.6mm agar layer before imaging experiments
of tissue mimicking phantoms with tumor cells.

Our findings show that antibodies labeled with AF750 and AF700 exhibited significantly higher diffusion coefficients than those labeled with IR680 and IR800. This difference suggests the physicochemical properties of the fluorophores, particularly molecular weight because smaller molecules diffuse more rapidly and hydrophilicity substantially influences diffusion behavior in the agarose environment. 
These variations directly affect experimental timing, as faster diffusion results in shorter times to reach equilibrium.

For mPAI, the calculated diffusion times for the various agents ranged from approximately 55 to 100 minutes. For PAI with soluble PD1, 
these times were shorter, ranging from 34 to 42 minutes. This difference is attributed to the larger molecular weights of the antibodies used in mPAI compared to the proteins and free dyes used in PAI with soluble PD1. To ensure sufficient binding and to account for slight variations between agents, we standardized the incubation period: 120 minutes for mPAI and 60 minutes for PAI with soluble PD1. This buffered approach ensures that all agents have ample time to reach binding equilibrium before imaging, minimizing the risk of dye wash-off before all imaging agents fully diffuse through the 10.6-mm agar layer.

These findings laid the groundwork for utilizing tissue mimicking phantom with tumor cells to evaluate the binding potential of various imaging agents. Importantly, the ability to capture real-time differences in diffusion and binding supports the broader aim of developing a dynamic platform to quantify PDL1 expression and availability using mPAI and PAI with soluble PD1. This approach advances
the potential for imaging-based biomarkers to improve the prediction of positive response to immune checkpoint inhibitor (ICI) treatments.
