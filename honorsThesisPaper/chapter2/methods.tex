\textit{Agarose-based tissue mimicking phantoms protocol}

To prepare the tissue phantom, $0.3$ g of agarose powder was mixed with $50$ mL of phosphate buffered saline (PBS) and heated on a hot plate while stirring until the agarose was fully dissolved, yielding a $0.6\% (w/v)$ agarose solution. The solution was then cooled to approximately $39^\circ C$ before combining $10$ mL of the agarose solution with $10$ mL of warm PBS to create a mixture of agarose $0.3\% (w/v)$. A volume of $5000$ µL of the $0.3\% (w/v)$ agarose mixture was added to each well of a dark-colored, four-well $3D$-printed plate, with each well having a diameter of $35.34$ mm and a 
depth of $19.10$ mm. The plate was placed in a digital dry bath and incubated for $15-30$ minutes to allow the agarose to solidify. After gelation, a central hole was made in the agarose of each well using a $16$-gauge $\times \frac{1}{2}"$ Luer stub needle. A background fluorescence image was acquired prior to dye introduction. Then, $12$ µL of fluorescent dye was placed in the hole. After dye introduction, fluorescence images were acquired at $5$-minute intervals over a $60$-minute period. 
Background images were used for noise subtraction during image analysis.