For the PAI experiment utilizing soluble PD1, it was necessary to identify two fluorescent dyes with spectrally distinct emission profiles suitable for concurrent detection. To guide the selection of appropriate dye pairs, we conducted a series of liquid phantom experiments designed to assess both the linearity of the fluorescence signal and the degree of spectral interference between fluorophores. 

Liquid phantoms are optically tunable models designed to mimic the absorption and scattering properties of biological media. In this study, liquid phantoms were employed to evaluate the spectral compatibility of our four fluorescent dyes, ensuring the selection of two dyes with minimal signal overlap and enabling accurate quantification of individual receptor concentrations. To validate dye selection, 
we first conducted linearity tests for each dye individually, followed by combination experiments to assess potential spectral interference. These procedures will be discussed in detail in Section ~\ref{sec:chapter3_methods}.

Linearity tests were performed to determine whether the intensity of the fluorescence signal increased proportionally with the concentration of the dye. This allowed us to identify the linear response range for each fluorophore. Based on the results of the linearity test, we selected a single reference signal
intensity that fell within the linear range for all dyes. This signal value was then used to determine the corresponding concentration of each dye in the combination experiments.