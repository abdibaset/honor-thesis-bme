The liquid phantom experiments enabled quantitative assessment of fluorescence linearity and spectral compatibility across four candidate dyes for use in PAI. As shown in Figure ~\ref{fig:linearity_test_plots}, each dye demonstrated a linear relationship between signal intensity and concentration in the experimental range. These linearity data were subsequently employed to determine appropriate dye concentrations for multi-dye combination experiments.

In the combination experiments, analysis of the 700 nm channel revealed significant differences in signal intensity between individual dyes AF700 and IR680 and their combinations with AF750 and IR800. However, the observed difference in signal for IR680 was less significant than for AF700, indicating reduced cross-talk when IR680 is combined with AF750 or IR800. Consequently, IR680 was selected for the 700 nm channel to serve as our untargeted agent. 

For the 800 nm channel, no significant differences in signal intensity were observed between AF750 and IR800, and their combinations with IR680 and AF700. However, as illustrated in Figure ~\ref{fig:linearity_test_plots_f}, the linearity tests showed that IR800 contributed less signal to the 700 nm channel compared to AF750, indicating reduced spectral bleed-through into the 700 nm channel. This reduced spectral interference led to the selection of IR800 for the 800 nm channel, designated for conjugation to soluble PD1. These results are summarized in Figures ~\ref{fig:combo_700} and ~\ref{fig:combo_800}.
