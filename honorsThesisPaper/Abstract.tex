\pagestyle{plain}
\begin{center}


\section*{Abstract}


\end{center}

Immune checkpoint inhibitors (ICIs) targeting the programmed cell death protein 1 (PD1) and its ligand, programmed death-ligand 1 (PDL1), have transformed cancer immunotherapy by disrupting inhibitory signaling pathways. However, only a subset of patients achieve durable responses. This limitation is in part due to current diagnostic approaches that rely on static immunohistochemical (IHC) assessments of PDL1 expression, which fail to capture its dynamic and functional availability. In addition, PDL1 can become unavailable for PD1 binding through \textit{cis}-interactions with CD80, further complicating its reliability as a predictive biomarker.

To address these challenges, this work develops and validates a paired-agent imaging (PAI) approach using fluorescence-based imaging to dynamically quantify both the expression and functional availability of PDL1 \textit{in vitro}. A multi-spectral PAI (mPAI) method was first implemented to measure the total expression of the PD1, PDL1, and CD80 receptors using four spectrally distinct antibody–fluorophore conjugates. Diffusion coefficients and corresponding times were experimentally determined in cell-free tissue-mimicking phantoms and used to establish incubation times for imaging in tissue mimicking phantoms containing tumor cells.

A second project focused on quantifying the fraction of PDL1 available for PD1 binding using soluble PD1 protein. The binding potential was measured under both unstimulated and interferon-gamma (IFN-$\gamma$)–stimulated conditions in lymphoma cells. Results demonstrated that IFN-$\gamma$ stimulation increased both PDL1 expression thus the binding potential, indicating increased receptor availability. Collectively, these methods enable the distinction between total PDL1 expression and its functional accessibility for PD1 engagement, providing a refined framework for evaluating ICI response and advancing the development of imaging-based biomarkers.

\cleardoublepage
