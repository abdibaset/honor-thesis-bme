QFC analysis demonstrated that PDL1 expression increased in lymphoma cells following IFN-$\gamma$ treatment. This observation is consistent with previous studies indicating that IFN-$\gamma$ upregulates PDL1 expression through activation of the JAK/STAT signaling pathway \cite{garciadiaz2017interferon, mitsuiki2019ctla}. Notably, CD80 expression was also upregulated under IFN-$\gamma$ stimulation. However, only a subset of PDL1 molecules likely formed \textit{cis}-dimers with CD80, leaving a portion of PDL1 available for trans-binding with PD1. This interpretation is supported by findings from Zhao et al. (2019), who reported that a 3.5-fold molar excess of CD80 is required to block approximately $80\%$ of PDL1:PD1 interactions via cis-binding \cite{zhao2019pdl1cd80}. These results suggest that while CD80 may partially sequester PDL1, a functionally accessible PDL1 pool likely remains under physiological conditions.
    
Unexpectedly, the highest binding potential was observed in the control condition (2.342), which lacked cells and was therefore expected to exhibit minimal or no specific binding. This result likely reflects nonspecific accumulation of the targeted or untargeted imaging agents, potentially due to incomplete removal of the dye during PBS rinsing. The IFN-$\gamma$-treated condition exhibited a higher binding potential than the untreated condition, suggesting an increased availability of PDL1 in IFN-$\gamma$-treated cells. This finding is further corroborated by the corresponding QFC data. These findings emphasize the importance of distinguishing between receptor expression and availability when evaluating immune checkpoint interactions using molecular imaging approaches.
