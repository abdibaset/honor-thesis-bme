After quantification of PDL1 expression in lymphoma tumor cells using mPAI, the next objective was to determine the fraction of PDL1 available for binding to PD1. PDL1 does not only interact in \textit{trans} with PD1, it can also form \textit{cis}-dimers with CD80 on the same cell surface \cite{sugiura2019restriction}. This \textit{cis}-interaction sequesters PDL1 and renders it functionally
unavailable for PD1 binding. To quantify the available PDL1, that is, the fraction capable of binding PD1, we used a PAI approach. In this experiment, the targeted agent consisted of a fluorescently labeled soluble PD1 protein, while the untargeted agent was a spectrally distinct free dye. 

As in the mPAI experiments, the accurate quantification depends on the careful selection of dyes with compatible optical and detection properties. To this end, preliminary liquid phantom experiments were conducted to identify two fluorophores suitable for simultaneous imaging. Linearity 
assessments were performed to determine the concentration ranges over which each dye exhibited a linear fluorescence response. Subsequently, dye combination experiments were conducted to evaluate the spectral crosstalk between the $700$ nm and $800$ nm imaging channels. The combination
experiments enabled the selection of dye pairs that could be imaged concurrently without compromising signal specificity. 

After the selection of suitable dye pairs, the next step involved determining the optimal incubation times for these selected pairs with their respective conjugated proteins. To achieve this, we conducted diffusion experiments utilizing tissue-mimicking phantoms without cells for both
the targeted and untargeted agents. This was followed by tissue-mimicking phantom experiments with lymphoma cells. To quantify the available PDL1 receptors for PD1 binding, a binding potential analysis was performed. 