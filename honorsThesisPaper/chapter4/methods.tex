\textit{Agarose with cells tissue mimicking phantoms}

To prepare tissue mimicking phantoms, $0.3$ g of agarose powder was dissolved in 50 mL of PBS by heating the mixture on a hot plate with continuous stirring until agarose was fully solubilized, resulting in a $0.6\% (w/v)$ agarose solution. The solution was then cooled to approximately $39^\circ C$ before use. Three experimental conditions were prepared: a control condition in which $0.6\% (w/v)$ agarose was combined with $500$ µL of PBS without any cells; an unstimulated cell condition in which 500 µL agarose solution was mixed with 500 µL of PBS containing 25 million lymphoma cells; and a stimulated condition in which 25 million  Lymphoma cells were pre-treated with interferon-gamma (IFN-$\gamma$) for 24 hours, suspended in $500$ µL of PBS, and mixed with 500 µL of the agarose solution. Before the introduction of imaging agents, a background fluorescence image was acquired for each well. Then, $1000$ µL of solution, comprising a mixture of the targeted agent, untargeted agent, and PBS, was added to each well and incubated for a predetermined incubation time of $60$ minutes. Following incubation, an image was captured, after which the phantom was rinsed with PBS for 5 minutes. The supernatant was removed and another image was taken. This rinsing process was repeated 15 times to monitor the clearance and agent retention within the phantom over time. 

To analyze the cells shown in Figures ~\ref{fig:tissue-cells_a} and ~\ref{fig:tissue-cells_b}, a circular ROI was drawn using MATLAB, and the resulting ROI was analyzed with the rinsing paired agent model (rpam) algorithm developed by Xu et al. \cite{xu2017rpam}

To determine the expression levels of PDL1, PD1, and CD80 under these conditions, a quantitative flow cytometry (QFC) assay was performed in parallel with the phantom experiments. This served to validate receptor expression profiles in stimulated and unstimulated tumor cells.

