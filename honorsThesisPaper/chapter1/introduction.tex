Programmed cell death protein 1 (PD1) is an immune checkpoint receptor expressed on activated T cells. 
It functions as an inhibitory receptor, helping to limit excessive immune responses and maintain
immune tolerance. PD1 binds to its ligand, programmed death-ligand 1 (PDL1), which is expressed in antigen-presenting cells (APCs) and many other cell types, including normal and tumor cells. This interaction sends an inhibitory signal to the T cell, reducing its activity and preventing potential collateral tissue damage or autoimmunity. Tumor cells can exploit this mechanism by expressing PDL1, falsely signaling to activated T cells that they are normal cells. This evasion not only facilitates tumor survival but also contributes to T-cell exhaustion and dysfunction
\cite{dong2002tumor, freeman2000engagement, liu2024pdl1cis, topalian2015immune}. 

Immune checkpoint inhibitors (ICI), such as anti-PD1 and anti-PDL1 monoclonal antibodies, pharmacologically block the PD1:PDL1 interaction. Despite their success in improving overall survival in several difficult-to-treat cancers, only a minority of patients experience durable clinical benefits
\cite{bruni2020immunoscore, haslam2020estimation}. Clinically, anti-PD1 therapy is often recommended based on tumor PDL1 expression levels \cite{galon2019approaches}. Currently, immunohistochemical staining (IHC) of primary tumors is the standard diagnostic method for assessing PDL1 expression. However, IHC provides only a static snapshot and does not capture the dynamic nature of PDL1 availability. PDL1 expression is conditional, often transient, and is induced by local immune activity, particularly T cell infiltration. As a result, IHC can misclassify tumors as negative for PDL1 if inflammation is absent at the time of biopsy \cite{ribas2016pdl1}.

Recently,  it has been shown that PDL1 can also \textit{cis}-dimerize with CD80 on the same cell surface. 
This \textit{cis}-interaction reduces the amount of PDL1 available for PD1 binding, potentially impairing the efficacy of therapies targeting this pathway \cite{chaudhri2018pdcd80, liu2024pdl1cis, sugiura2019restriction}. As a result, surface PDL1 levels alone may not reflect the functional availability of PDL1, complicating its use as a predictive biomarker for the ICI response. 

Given the limitations of static diagnostic methods like immunohistochemistry, there is a critical need to develop a dynamic approach to assess the engagement of PDL1 with both PD1 and CD80. Paired agent imaging (PAI) has previously been demonstrated as a quantitative fluorescence imaging technique to measure receptor availability \textit{in vivo} \cite{tichauer2015review}. Building on this approach, the goal of this research is to develop an \textit{in vivo} optical fluorescence imaging technique capable of simultaneously quantifying the expression of PD1, PDL1, and CD80, as well as determining the fraction of PDL1 available for PD1 binding. The methodology is described in detail below.