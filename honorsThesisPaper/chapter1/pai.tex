PAI is a fluorescence-based molecular imaging technique that involves the simultaneous administration 
of a receptor-targeted imaging agent and an untargeted control agent, each conjugated to spectrally 
distinct fluorescent dyes but exhibiting matched pharmacokinetic profiles. By enabling the separation 
of specific binding from non-specific uptake, PAI's computation model facilitates the quantitative 
estimation of \textit{in vivo} receptor concentration and availability.

To evaluate immune checkpoint receptor expression, we developed a multi-spectral paired-agent 
imaging (mPAI) approach capable of simultaneously quantifying the concentrations of PD1, PDL1, and CD80. 
This methodology involved the co-administration of four monoclonal antibodies: three receptor-targeted 
antibodies—anti-PD1, anti-PDL1, and anti-CD80 and one untargeted isotype control (IgG). Each antibody 
was conjugated to a unique fluorophore, enabling multiplexed, spectrally resolved quantification of 
receptor expression.

Following the quantification of PDL1 expression, a two-agent PAI experiment was conducted to assess 
receptor availability. This experiment employed a fluorescently labeled soluble PD1 protein as the 
targeted agent and a spectrally distinct free dye as the untargeted control. Receptor availability 
was quantified via calculation of the binding potential, reflecting the accessible fraction of PDL1 
capable of engaging PD1.

The PAI experiments utilized near-infrared fluorescent dyes: IRDye680LT (IR680), Alexa Fluor 700 (AF700), 
Alexa Fluor 750 (AF750), and IRDye800CW (IR800). Imaging was performed using the Pearl® Imaging System, 
which incorporates two optical detection channels centered at 700 nm and 800 nm. IR680 and AF700 were 
detected in the 700 nm channel, while AF750 and IR800 were detected in the 800 nm channel, ensuring 
minimal spectral overlap, and enabling accurate fluorophore discrimination.
