This thesis demonstrates the successful development and application of two distinct PAI methodologies to address critical limitations in assessing immune checkpoint dynamics. First, we established the feasibility of an mPAI framework for the quantitative assessment of total receptor expression for PD1, PDL1, and CD80. Second, we developed a PAI approach utilizing soluble PD1 to specifically quantify the fraction of PDL1 receptors that are functionally available for PD1 binding, a key determinant for the efficacy of immune checkpoint inhibitor (ICI) therapies.

Our findings reveal that IFN-$\gamma$ treatment up-regulates total PDL1 expression, as confirmed by QFC. The PAI experiments distinctly showed a higher binding potential in IFN-$\gamma$-treated conditions compared to untreated cells, directly reflecting an increased available PDL1 pool. By rigorously distinguishing between total receptor expression and functionally accessible receptor populations, this research provides a more dynamic framework for evaluating immune checkpoint interactions. 
