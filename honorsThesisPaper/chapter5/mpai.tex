The successful development and validation of the tissue mimicking phantom model, along with the determination of the diffusion coefficients, have demonstrated the feasibility of the mPAI approach \textit{in vitro}. Based on these findings, future work should focus on translating this imaging methodology to \textit{in vivo} models. Specifically, implementing mPAI in murine tumor models will allow the physiological validation of receptor expression patterns and agent behavior in living tissue. Such \textit{in vivo} studies will be instrumental in evaluating the diagnostic potential of mPAI under conditions that reflect the complex tissue architecture and actual tumors. Furthermore, longitudinal imaging could offer insight into dynamic changes in receptor availability over time, further enhancing the utility of mPAI to evaluate immune checkpoint therapy responses.