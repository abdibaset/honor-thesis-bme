In the combination liquid phantom experiments, we observed that the signal intensity of AF700 decreased when combined with AF750 or IR800, despite theoretical expectations that the combined signal would remain constant or increase. This unexpected reduction suggests possible dye–dye interactions, quenching effects, or optical interference. Future studies should explore the physicochemical or spectral interactions between AF700 and other fluorophores.

Furthermore, in tissue-mimicking phantoms, the control condition, which lacked cells, showed an unexpectedly high binding potential of 2.342. In the absence of cellular receptors, the soluble PD1 agent was not expected to bind specifically, and the binding potential should theoretically be zero. This discrepancy suggests possible nonspecific retention of the imaging agents or incomplete removal of unbound dye during the washing steps. Future experiments should implement more rigorous washing protocols and include replicate trials under the same conditions.

To further validate the generalizability of our findings, future studies could also explore PDL1 dynamics across different tumor cell lines, verifying whether the observed binding potential trends are consistently replicated.
